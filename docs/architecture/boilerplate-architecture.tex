\documentclass{article}

\usepackage[colorlinks]{hyperref}
\usepackage{booktabs}
\usepackage{graphicx}
\usepackage[svgnames]{xcolor}
\usepackage{fancyvrb}

\input{boxes}
\input{commands}

\title{Streamlined Application Development\\with \boil}
\author{Dr.\ Tom Nurkkala}

\begin{document}
\maketitle
\tableofcontents

\section{Introduction}
\label{sec:introduction}

Many components of a database-backed web application
depend critically
on the specifics of the database schema.
For example:
\begin{enumerate}
\item Object-Relational Mapper (ORM) model classes
\item Data transfer objects (DTO's)
\item Database service classes
\item RESTful API resources
\item GraphQL object types, input types, and resolvers
\item UI CRUD views that maintain database content
\end{enumerate}
Creating
and updating
such components manually
can be tedious, inconsistent, and error prone.
Different components
rely on different aspects
of the underlying database schema
and may employ different naming conventions and
coding practices.

\begin{figure}[h]
  \centering
  \includegraphics[width=\textwidth]{block-diagram}
  \caption{\boil{} block diagram}
  \label{fig:block-diagram}
\end{figure}

\end{document}

%%% Local Variables:
%%% mode: latex
%%% TeX-master: t
%%% End:

% LocalWords:  DTO's GraphQL
